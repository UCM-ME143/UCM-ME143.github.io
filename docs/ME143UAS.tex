% Options for packages loaded elsewhere
\PassOptionsToPackage{unicode}{hyperref}
\PassOptionsToPackage{hyphens}{url}
%
\documentclass[
]{book}
\usepackage{lmodern}
\usepackage{amssymb,amsmath}
\usepackage{ifxetex,ifluatex}
\ifnum 0\ifxetex 1\fi\ifluatex 1\fi=0 % if pdftex
  \usepackage[T1]{fontenc}
  \usepackage[utf8]{inputenc}
  \usepackage{textcomp} % provide euro and other symbols
\else % if luatex or xetex
  \usepackage{unicode-math}
  \defaultfontfeatures{Scale=MatchLowercase}
  \defaultfontfeatures[\rmfamily]{Ligatures=TeX,Scale=1}
\fi
% Use upquote if available, for straight quotes in verbatim environments
\IfFileExists{upquote.sty}{\usepackage{upquote}}{}
\IfFileExists{microtype.sty}{% use microtype if available
  \usepackage[]{microtype}
  \UseMicrotypeSet[protrusion]{basicmath} % disable protrusion for tt fonts
}{}
\makeatletter
\@ifundefined{KOMAClassName}{% if non-KOMA class
  \IfFileExists{parskip.sty}{%
    \usepackage{parskip}
  }{% else
    \setlength{\parindent}{0pt}
    \setlength{\parskip}{6pt plus 2pt minus 1pt}}
}{% if KOMA class
  \KOMAoptions{parskip=half}}
\makeatother
\usepackage{xcolor}
\IfFileExists{xurl.sty}{\usepackage{xurl}}{} % add URL line breaks if available
\IfFileExists{bookmark.sty}{\usepackage{bookmark}}{\usepackage{hyperref}}
\hypersetup{
  pdftitle={ME143 - Unmanned Aircraft Systems},
  pdfauthor={Dr.~Brandon Stark},
  hidelinks,
  pdfcreator={LaTeX via pandoc}}
\urlstyle{same} % disable monospaced font for URLs
\usepackage{longtable,booktabs}
% Correct order of tables after \paragraph or \subparagraph
\usepackage{etoolbox}
\makeatletter
\patchcmd\longtable{\par}{\if@noskipsec\mbox{}\fi\par}{}{}
\makeatother
% Allow footnotes in longtable head/foot
\IfFileExists{footnotehyper.sty}{\usepackage{footnotehyper}}{\usepackage{footnote}}
\makesavenoteenv{longtable}
\usepackage{graphicx,grffile}
\makeatletter
\def\maxwidth{\ifdim\Gin@nat@width>\linewidth\linewidth\else\Gin@nat@width\fi}
\def\maxheight{\ifdim\Gin@nat@height>\textheight\textheight\else\Gin@nat@height\fi}
\makeatother
% Scale images if necessary, so that they will not overflow the page
% margins by default, and it is still possible to overwrite the defaults
% using explicit options in \includegraphics[width, height, ...]{}
\setkeys{Gin}{width=\maxwidth,height=\maxheight,keepaspectratio}
% Set default figure placement to htbp
\makeatletter
\def\fps@figure{htbp}
\makeatother
\setlength{\emergencystretch}{3em} % prevent overfull lines
\providecommand{\tightlist}{%
  \setlength{\itemsep}{0pt}\setlength{\parskip}{0pt}}
\setcounter{secnumdepth}{5}
\usepackage{booktabs}
\usepackage[]{natbib}
\bibliographystyle{apalike}

\title{ME143 - Unmanned Aircraft Systems}
\author{Dr.~Brandon Stark}
\date{2020-09-24}

\usepackage{amsthm}
\newtheorem{theorem}{Theorem}[chapter]
\newtheorem{lemma}{Lemma}[chapter]
\newtheorem{corollary}{Corollary}[chapter]
\newtheorem{proposition}{Proposition}[chapter]
\newtheorem{conjecture}{Conjecture}[chapter]
\theoremstyle{definition}
\newtheorem{definition}{Definition}[chapter]
\theoremstyle{definition}
\newtheorem{example}{Example}[chapter]
\theoremstyle{definition}
\newtheorem{exercise}{Exercise}[chapter]
\theoremstyle{remark}
\newtheorem*{remark}{Remark}
\newtheorem*{solution}{Solution}
\begin{document}
\maketitle

{
\setcounter{tocdepth}{1}
\tableofcontents
}
\hypertarget{course-information}{%
\chapter{Course Information}\label{course-information}}

This is the course lecture material for UC Merced ME143. All course assignments, exams, quizzes and the syllabus can be found in Catcourses.

For information on how to compile or edit the course material, please visit the github repository.

\begin{center}\includegraphics[width=0.5\linewidth]{images/general/UCM_heart} \end{center}

\hypertarget{ch-intro}{%
\chapter{Introduction}\label{ch-intro}}

Introduction to UAS

\textbf{Daily Show with Jon Stewart (January 2013) - Interview with Missy Cummings - Rise of the Drones}

\begin{itemize}
\tightlist
\item
  \url{http://www.cc.com/video-clips/87aevs/the-daily-show-with-jon-stewart-exclusive---missy-cummings-extended-interview-pt--1?xrs=share_copy_email}
\item
  \url{http://www.cc.com/video-clips/rwkvkp/the-daily-show-with-jon-stewart-exclusive---missy-cummings-extended-interview-pt--2}
\item
  \url{http://www.cc.com/video-clips/91upnm/the-daily-show-with-jon-stewart-exclusive---missy-cummings-extended-interview-pt--3}
\end{itemize}

\hypertarget{language-and-terminology}{%
\section{Language and Terminology}\label{language-and-terminology}}

\begin{definition}[Unmanned Aircraft]
\protect\hypertarget{def:defUA}{}{\label{def:defUA} \iffalse (Unmanned Aircraft) \fi{} }A drone, remote-controlled pilotless aircraft, or a human-designed device that is used to intended to be used for flight in the air that is operated without the possibility of direct human intervention from within or on the aircraft.
\end{definition}

\begin{definition}[Unmanned Aircraft System]
\protect\hypertarget{def:defUAS}{}{\label{def:defUAS} \iffalse (Unmanned Aircraft System) \fi{} }An unmanned aircraft and all of its associated elements (including communication links and the components that control the small unmanned aircraft) that are required for the safe and efficient operation of the unmanned aircraft.
\end{definition}

\begin{definition}[Small Unmanned Aircraft]
\protect\hypertarget{def:defSUA}{}{\label{def:defSUA} \iffalse (Small Unmanned Aircraft) \fi{} }An unmanned aircraft that weighs less than 55 lbs on takeoff, including everything that is on board or otherwise attached to the aircraft.
\end{definition}

\begin{definition}[Small Unmanned Aircraft System]
\protect\hypertarget{def:defSUAS}{}{\label{def:defSUAS} \iffalse (Small Unmanned Aircraft System) \fi{} }A small unmanned aircraft and its associated elements (including communication links and the components that control the small unmanned aircraft) that are required for the safe and efficient operation of the small unmanned aircraft.
\end{definition}

\begin{corollary}[Model Aircraft]
\protect\hypertarget{cor:defMA}{}{\label{cor:defMA} \iffalse (Model Aircraft) \fi{} }In the current regulatory framework, there is no definition of a model aircraft or legal distinction between the flying toy you buy at Walmart, the RC plane from the hobby store or a 30 lb industrial drone. All are considered Unmanned Aircraft.
\end{corollary}

Unmanned Aircraft refers to just to the object that flies, but is usually only used when there is a need to make a distinction. The same with small UAS (sUAS) and UAS - it's usually only used when there needs to be an explicit distinction between drones under 55 lbs and those that are heavier than 55 lbs.

\hypertarget{some-examples-of-unmanned-aircraft-systems}{%
\subsection{Some examples of Unmanned Aircraft Systems}\label{some-examples-of-unmanned-aircraft-systems}}

\begin{figure}

{\centering \includegraphics[width=0.5\linewidth]{images/general/quadrotor} 

}

\caption{A regular quadrotor design}\label{fig:quad}
\end{figure}

\begin{figure}

{\centering \includegraphics[width=0.5\linewidth]{images/general/rc_plane} 

}

\caption{A remote control plane (rc plane)}\label{fig:rcplane}
\end{figure}

\begin{figure}

{\centering \includegraphics[width=0.5\linewidth]{images/general/military_drone} 

}

\caption{A military drone}\label{fig:militarydrone}
\end{figure}

\hypertarget{some-other-names-of-drones}{%
\subsection{Some other names of drones}\label{some-other-names-of-drones}}

UAV - unmanned aircraft vehicle

UAS - unmanned aerial system

RPAS - remotely piloted aircraft system

\begin{itemize}
\tightlist
\item
  Commonly used internationally
\end{itemize}

Multi-rotor and similar terms

\begin{itemize}
\tightlist
\item
  quadrotor, quadcopter
\item
  hexirotor, hexicopter
\end{itemize}

\begin{quote}
Helicopter is derived from `helico' (spiral/whirl) and `ptero' (wing as in pterodactyl) from Greek. Despite it's common use, `copter' isn't actually a valid suffix.
\end{quote}

\begin{quote}
Even worse, `quad' is a Latin prefix. So when you say `quadcopter' you're mixing a Latin prefix with a mangled Greek suffix. Truly an American English word.
\end{quote}

\hypertarget{common-drone-manufacturers}{%
\subsection{Common Drone Manufacturers}\label{common-drone-manufacturers}}

\begin{itemize}
\tightlist
\item
  DJI - (Da-Jiang Innovations) - \url{https://www.dji.com/}

  \begin{itemize}
  \tightlist
  \item
    Probably the most well-known and ubiquous non-military drone company
  \end{itemize}
\item
  Sensefly - Parrot Group - \url{https://www.sensefly.com/}

  \begin{itemize}
  \tightlist
  \item
    Most known for their eBee foam fixed-wing drone for mapping operations
  \end{itemize}
\end{itemize}

\hypertarget{history}{%
\section{History}\label{history}}

A short timeline of notable UAS history

\begin{itemize}
\tightlist
\item
  The first ``claimed'' heavier-than-air, powered flight was of a UAS over the Potomac River, by Dr.~Sam Langley (1896)
\item
  The first radio-controlled UAS flights were converted Navy trainers, flown over Long Island using the recently invented automatic gyroscopic stabilizer (1917)
\item
  The first mass production UAS was the Kettering Bug, costing \$400 and capable of carrying a 300 lb (1918) - History Channel clip - \url{https://youtu.be/gNO84yh2ZxY}
\item
  The British Queen Bee UAS was landed by dragging a weighted antenna line on the ground, which then pulled back the stick and the throttle to become the first re-usable UAS (1935)
\item
  Reginald Denny (actor and model aircraft enthusiast) signed a contract with the Navy to deliver TDD-1 (Target Drone Denny 1) in 1939

  \begin{itemize}
  \tightlist
  \item
    The first recorded instance of `drone' being associated with a UAS
  \end{itemize}
\end{itemize}

Where does the term `Drone' come from?

\hypertarget{pop-culture}{%
\section{Pop-Culture}\label{pop-culture}}

This page was published using \textbf{bookdown}\citep{R-bookdown} using RStudio in R Markdown and \textbf{knitr} \citep{xie2015}. The raw files can be found in the corresponding github page here:

\hypertarget{regulations}{%
\chapter{Regulations}\label{regulations}}

\hypertarget{us-regulations}{%
\section{US Regulations}\label{us-regulations}}

\hypertarget{airspace-charts}{%
\section{Airspace Charts}\label{airspace-charts}}

\hypertarget{issues-and-controversies}{%
\section{Issues and Controversies}\label{issues-and-controversies}}

\hypertarget{part-uas-modeling-and-control}{%
\part{UAS Modeling and Control}\label{part-uas-modeling-and-control}}

\hypertarget{uas-dynamics}{%
\chapter{UAS Dynamics}\label{uas-dynamics}}

\hypertarget{fixed-wing-models}{%
\section{Fixed-Wing Models}\label{fixed-wing-models}}

\hypertarget{rotary-wing-models}{%
\section{Rotary-Wing Models}\label{rotary-wing-models}}

We describe our methods in this chapter.

\hypertarget{refs}{}

\hypertarget{appendix-appendix}{%
\appendix}


\hypertarget{digital-systems}{%
\chapter{Digital Systems}\label{digital-systems}}

\hypertarget{vector-notation}{%
\chapter{Vector Notation}\label{vector-notation}}

  \bibliography{book.bib,packages.bib}

\end{document}
